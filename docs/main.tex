\documentclass[12pt, twoside, hidelinks, a4paper]{article}

\usepackage[]{geometry}
\geometry{inner=30mm, outer=20mm, top=23mm, bottom=23mm}

\usepackage{mystyle}
\pagestyle{headings}

\usepackage{fancyhdr}
\fancyhf{}
\pagestyle{fancy}
\renewcommand{\headrulewidth}{0pt}
% numery stron: lewa do lewego, prawa do prawego
\fancyfoot[LE,RO]{\thepage}

\fancypagestyle{plain}
{
   \fancyhf{}
\renewcommand{\headrulewidth}{0pt}
% numery stron: lewa do lewego, prawa do prawego
\fancyfoot[LE,RO]{\thepage}
}

\usepackage{pdfpages}
\usepackage{amsfonts}
%\renewcommand{\familydefault}{\sfdefault}
\setlength\parindent{1cm}

\usepackage{indentfirst}
\usepackage[affil-it]{authblk}
\usepackage{smartdiagram}
\usepackage{metalogo}
\usepackage{moreverb}

\let\lll\undefined
\usepackage{amssymb}

\begin{document}
    \setstretch{1.15}
 	\pagenumbering{arabic}

\author{Marcin Waszak}
\title{OWD -- sprawozdanie z projektu}
\date{5 grudnia 2018}
\affil{Wydział Elektroniki i Technik Informacyjnych, Politechnika Warszawska}


\maketitle

\begin{abstract}
Celem projektu jest optymalizacja pracy elektrowni pod względem kosztów jak i możliwości zaspokojenia wzrostu zapotrzebowań na energię ponad normę. Ponadto zostaną porównane techniki optymalizacji wielokryterialnej przy użyciu skalaryzacji \textit{Metodą Ważenia Ocen} (MWO) oraz \textit{Przedziałową Metodą Punktu Odniesienia} (PMPO).
\end{abstract}

\section{Zadanie}
Kod zadania: \textbf{OWD AK5}

Prowadzący projekt: dr inż. Adam Krzemienowski

\section{Analityczne sformułowanie problemu}
$P$ - uporządkowany zbiór okresów,

$T$ - uporządkowany zbiór typów generatorów (T1, T2, T3),

$N$ - uporządkowany zbiór generatorów danego typu,

Przyjmijmy: $p \in P$, $t \in T, i \in I[t]$

\subsection{Parametry}
\begin{itemize}
\item $periods\_demand \{P\}$
\item $periods\_length \{P\}$
\item $available\_generators \{T\}$
\item $load\_min \{T\}$
\item $load\_max \{T\}$
\item $cost\_min \{T\}$
\item $cost\_linear \{T\}$
\item $cost\_start \{T\}$
\item $demand\_total = \sum_{p}^{p \in P} periods\_demand[p]$
\end{itemize}

\subsection{Zmienne}
\begin{itemize}

%\item $\forall p \forall t \forall i : active{P,T,I} binary$
\item $active{P,T,I} binary$
\item $load{P,T,I}$
\item $production \; {p \in P} = \sum_{t}^{t \in T} \sum_{i}^{i \in I[t]} \; load[p,t,i];$
\item $production\_total = \sum_{p}^{p \in P} \; production[p]$
\item $demand\_increase = \frac{production\_total}{demand\_total}$
\item $toggled \{p \in P, t \in T, i \in I[t] \} = active[p,t,i] - active[((p+3) mod |P|)+1,t,i]$
\item $started \{p \in P, t \in T, i \in I[t] \} = \frac{toggled[p,t,i]^2 + toggled[p,t,i]}{2}$
\item $cost\_launch \{p \in P, t \in T, i \in I[t] \} = cost\_start[t] * started[p,t,i]$
\item $cost\_usage \{p \in P, t \in T, i \in I[t] \} = cost\_min[t] + cost\_linear[t] * (load[p,t,i] - load\_min[t])$
\item $cost\_total = \sum_{p}^{p \in P} \; \sum_{t}^{t \in T} \; \sum_{i}^{i \in I[t]} \; (periods\_length[p] * cost\_usage[p,t,i] + cost\_launch[p,t,i])$
\end{itemize}


\subsection{Ograniczenia}
\begin{itemize}
%st1
\item $\forall p \in P \; \forall t \in T \; \forall i \in I[t] \; : load[p,t,i] 
\geqslant load\_min[t] * active[p,t,i]$
%st2
\item $\forall p \in P \; \forall t \in T \; \forall i \in I[t] \; : load[p,t,i] \leqslant load\_max[t] * active[p,t,i]$
%st3
\item $sum\_generators \{ p \in P, t \in T \} = \sum_{i}^{i \in I[t]} active[p,t,i]$

$\forall p \in P : sum\_generators[p,T1] \leqslant sum\_generators[p,T2] + sum\_generators[p,T3]$
%st4
\item $\forall p \in P : \sum_{t}^{t \in T} \; \sum_{i}^{i \in I[t]} load[p,t,i] >= periods\_demand[p]$
\end{itemize}

\subsection{Funkcje celu}
\subsubsection{Funkcja celu dla minimalizacji kosztu}
Zapis minimalizacji kosztu jest trywialny:
$$min \: \: \leftarrow cost\_total$$

\subsubsection{Funkcja celu metody ważenia ocen}
Załóżmy, że znamy wartość wagi $weight$, to wtedy funkcja celu wygląda następująco:
$$max \: \: \leftarrow weight*demand\_increase - cost\_total$$
Należy zwrócić uwagę, że $cost\_total$ jest poprzedzone minusem, ponieważ kanonicznie metoda ważenia ocen jest problemem maksymalizacji. My natomiast koszt całkowity chcemy minimalizować.

\subsubsection{Funkcja celu dla przedziałowej metody punktu odniesienia}
Załóżmy, że znamy wartości parametrów $\epsilon$, $\gamma$, $\beta$, $a_1$, $r_1$, $a_2$, $r_2$. Wprowadźmy zmienne pomocnicze $v$, $z_1$, $z_2$. Ostatecznie przedziałowa metoda punktu odniesienia może być w zaimplementowana formułując następujące ograniczenia:
\begin{itemize}
\item $v \leqslant z_1$
\item $v \leqslant z_2$
\item $z_1 \leqslant \gamma * \frac{cost\_total - r_1}{a_1 - r_1}$
\item $z_1 \leqslant \frac{cost\_total - r_1}{a_1 - r_1}$
\item $z_1 \leqslant \beta * \frac{cost\_total - a_1}{a_1 - r_1} + 1$
\item $z_2 \leqslant \gamma * \frac{demand\_increase - r_2}{a_2 - r_2}$
\item $z_2 \leqslant \frac{demand\_increase - r_2}{a_2 - r_2}$
\item $z_2 \leqslant \beta * \frac{demand\_increase - a_2}{a_2 - r_2} + 1$
\end{itemize}
Wtedy funkcja celu ma następującą postać:
$$max \: \: \leftarrow v + \epsilon * (z_1 + z_2)$$

\begin{figure}[H]
\centering
\includegraphics[scale=0.35]{plot.png}
\caption{Przestrzeń rozwiązań efektywnych w przestrzeni względnej produkcji i kosztu}
\label{fig:plot}
\end{figure}

%\printbibliography

\end{document}
